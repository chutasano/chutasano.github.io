\documentclass[11pt]{article}
\usepackage[a4paper,margin=1in]{geometry}
\usepackage{enumitem}
\usepackage[colorlinks=true, urlcolor=blue, citecolor=black]
{hyperref}
\usepackage{titlesec}
\usepackage{fancyhdr}
\usepackage{url}
\usepackage[maxbibnames=15,sorting=ydnt, defernumbers=true]{biblatex}

\renewbibmacro*{doi+eprint+url}{%
\iftoggle{bbx:url}{\iffieldundef{doi}
{\usebibmacro{url+urldate}}{}}{}%
\newunit\newblock\iftoggle{bbx:eprint}{\usebibmacro{eprint}}{}%
\newunit\newblock\iftoggle{bbx:doi}{\printfield{doi}}{}}

% TODO need to manually maintain, should fix w/ longer-term solution
\DeclareBibliographyCategory{arxiv}
\addtocategory{arxiv}{Sano23arxiv, Sano21arxiv}

% Header
\pagestyle{fancy}
\fancyhf{}
\rhead{Chuta Sano}
\lhead{Curriculum Vitae}
\rfoot{\thepage}

\bibliography{cs}

% Section formatting
\titleformat{\section}{\large\bfseries}{}{0em}{}[\titlerule]

\begin{document}

\begin{center}
    {\LARGE \textbf{Chuta Sano}} \\
    PhD Student, McGill University \\
    \href{mailto:chuta.sano@mail.mcgill.ca}{chuta.sano@mail.mcgill.ca} \\
    Phone: 1-508-598-2492 \\
    Website: \href{https://chutasano.github.io}{https://chutasano.github.io}
\end{center}

\section*{Education}
\begin{itemize}[leftmargin=*]
    \item \textbf{McGill University}, PhD in Computer Science \\
    August 2022 – Present (Expected graduation: Summer 2026) \\
    Advisor: Prof. Brigitte Pientka
    \item \textbf{Carnegie Mellon University}, MS in Computer Science \\
    August 2018 – December 2019 \\
    Advisor: Prof. Frank Pfenning \\
    Thesis: On Session-Typed Contracts for Imperative Languages
    \item \textbf{University of Massachusetts Lowell}, BSc in Computer Science, Data Science Concentration, Math Minor \\
    August 2015 – May 2018 \\
    Advisor: Prof. Xinwen Fu \\
    Thesis: Turning Legacy IR Devices into Smart IoT Devices
\end{itemize}

\section*{Publications}
\printbibliography[heading=none, nottype=software, notcategory=arxiv]

\section*{Research}
\begin{itemize}[leftmargin=*]
  \item \textbf{McGill University}, August 2022 -- Present
    \begin{itemize}
      \item I investigated techniques to mechanize substructural languages via higher-order abstract syntax in the proof assistant Beluga. One approach~\cite{Sano23oopsla} led to a mechanization of session types, which have a linear type system. I also adviced many student projects that used this technique. I also worked on an alternate approach that later led to a general framework for mechanizing a whole suite of substructural languages~\cite{Zackon25cpp}.
      \item I am investigating language interoperability between languages with substantially different semantics, e.g., functional programming and session-typed process calculi. I develop a notion of first-class code that enables a functional language to generate, analyze, and run process calculus code~\cite{Sano25icfp}. I am generalizing this approach to capture interactions between functional languages and a broader set of languages, including an assembly-like language and a quantum programming language.
    \end{itemize}
  \item \textbf{Max Planck Institute for Software Systems}, September 2023 -- October 2023
    \begin{itemize}
      \item As a visiting researcher, I worked on extending session types with probability (no publications) and metaprogramming~\cite{Sano25icfp}.
    \end{itemize}
  \item \textbf{Carnegie Mellon University} August 2018 -- October 2021
    \begin{itemize}
      \item I extended the CC0 compiler, a C-like language extended with session-typed concurrency, with monitors. I extended the theory of monitors to support shared session types and formalized it in my thesis~\cite{Sano19ms}
      \item I extended shared session types with subtyping~\cite{Sano21coordination}, enabling the static encoding of shared protocols with ``phases,'' i.e., protocols that can be encoded in a (finite) state machine.
    \end{itemize}
  \item \textbf{University of Massachusetts Lowell}, \emph{Robotics Research Group} September 2016 -- May 2017
    \begin{itemize}
      \item I developed a ROS-based service that receives a top-down image of a robot and its surroundings to simulate laser scan outputs done by the robot. This was used in conjunction with TinyRobo, a swarm robot project.
      \item I implemented a real time FaceDetector for GoogleGlass using OpenCV and Android NDK, and then optimized it by devising a handoff mechanism between generic Android devices through first class continuations.
    \end{itemize}
  \item \textbf{University of Massachusetts Lowell}, \emph{Center for Internet Safety and Forensic Education and Research} August 2015 -- May 2018
    \begin{itemize}
      \item I surveyed various consumer phones with stereocameras and conducted experiments to determine their attack capabilities~\cite{Li17icc}.
      \item I researched common IR-based products and their protocols. This led to a proposing of a few attack models and the desigining of low-cost devices that bridge IR and IoT devices~\cite{Sano18wasa}.
      \item I maintained and enhanced the Privacy Enhancing Keyboard application for Android based on usability experiments~\cite{Ling17wasa, Ling17wcmc}.
    \end{itemize}
\end{itemize}

\section*{Conference/Invited Talks}
\begin{itemize}[leftmargin=*]
    \item \textbf{Fusing Session-Typed Concurrent Programming into Functional Programming}
    \begin{itemize}
        \item International Conference on Functional Programming (ICFP), October 2025
        \item McGill University (COMEPLS), September 2024
        \item Languages and Logic Montreal (LLM), March 2025
        \item Eastern Canada Logic and Programming Seminar (ECLaPS), December 2024
    \end{itemize}

    \item \textbf{Mechanizing Session-Types using a Structural View: Enforcing Linearity without Linearity}
    \begin{itemize}
        \item Object-Oriented Programming, Systems, Languages, and Applications (OOPSLA), October 2023
        \item University of Tokyo, May 2023
        \item Kyoto University, May 2023
        \item Eastern Canada Logic and Programming Seminar (ECLaPS), March 2023
    \end{itemize}

    \item \textbf{Manifestly Phased Communication via Shared Session Types}
    \begin{itemize}
      \item International Conference on Coordination Models and Languages (COORDINATION), June 2021
        \item McGill University (COMEPLS), October 2024
        \item University of Edinburgh, October 2021
    \end{itemize}

\end{itemize}




\section*{Teaching}
\begin{itemize}[leftmargin=*]
  \item McGill University
    \begin{itemize}
      \item Winter 2025: COMP527 – Logic and Computation, Teaching Assistant
      \item Winter 2023: COMP527 – Logic and Computation, Teaching Assistant
      \item Fall 2022: COMP302 – Programming Languages and Paradigms, Teaching Assistant
    \end{itemize}
  \item Oregon Programming Languages Summer School, 2021
    \begin{itemize}
      \item Course assistant: \emph{Session-Typed Concurrent Programming}.
    \end{itemize}

  \item University of Massachusetts Lowell
    \begin{itemize}
      \item Mathematics Tutor: Taught Calculus, Discrete Math, Linear Algebra, Probability, and Number Theory
    \end{itemize}
\end{itemize}

\section*{Services}
\begin{itemize}[leftmargin=*]
  \item Organizer of Conference of McGill's Epic Programming Language Systems (COMEPLS), 2023 -- 2025
  \item External Reviewer for FSCD, 2024
  \item External Reviewer for LFMTP, 2024
  \item Mentor for McGill CSGS Bridge Mentorship Program, 2023 -- 2024
\end{itemize}


\section*{Awards \& Honors}
\begin{itemize}[leftmargin=*]
    \item \textbf{Tomlinson Graduate Fellowship}, McGill University, 2022 -- 2025
    \item \textbf{Best Teaching Assistant Award}, McGill University, Winter 2023 \\
    Recognized for my TA-ship of COMP527 based on student nominations
  \item \textbf{Latorre Family Scholarship}, University of Massachusetts Lowell, 2017
  \item \textbf{Co-op Scholar}, University of Massachusetts Lowell, 2016
  \item \textbf{Dean's Scholarship}, University of Massachusetts Lowell, 2015 -- 2018
  \item \textbf{Professor George Grinstein Scholarship}, University of Massachusetts Lowell, 2015
\end{itemize}

\section*{Software Artifacts}
\nocite{*}
\printbibliography[heading=none, type=software, resetnumbers=true]




\end{document}
